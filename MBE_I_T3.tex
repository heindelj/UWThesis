\begin{table}[t]
\begin{adjustbox}{width=\columnwidth,center}
\begin{tabular}{@{}ccccc@{}}
\toprule
Cluster & 1-Body          & 2-Body           & Sum of 3- to 5-Body & Total Correlation \\ \midrule
\ce{(H2O)7}  & -3.73 (15.80)   & -19.77 (83.74)   & -0.105 (0.44)       & -23.609           \\
\ce{(H2O)_{10}} & -6.391 (16.59)  & -31.870 (82.75)  & -0.250 (0.65)       & -38.514           \\
\ce{(H2O)_{13}} & -9.135 (23.99)  & -29.0526 (76.31) & 0.135 (-0.35)       & -38.073           \\
\ce{(H2O)_{16}} & -12.051 (24.80) & -36.205 (74.49)  & -0.344 (0.71)       & -48.600           \\ \bottomrule
\end{tabular}
\end{adjustbox}
\begin{spacing}{1.0}
\caption[Contribution of the correlation energy to the various many-body terms for \ce{(H2O)n}, n = 7, 10, 13, 16. The results are obtained with the largest basis set used for each isomer (see text). The percentage of the total correlation energy is given in parentheses. Energies are in kcal/mol, while parentheses indicate the percentage of the total correlation energy.]{Contribution of the correlation energy to the various many-body terms for \ce{(H2O)n}, n = 7, 10, 13, 16. The results are obtained with the largest basis set used for each isomer (see text). The percentage of the total correlation energy is given in parentheses. Energies are in kcal/mol, while parentheses indicate the percentage of the total correlation energy.}\label{tab:MBE_I_T3}
\end{spacing}
\end{table}