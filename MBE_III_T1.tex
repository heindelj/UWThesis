\begin{table}[t]
\begin{adjustbox}{width=\columnwidth,center}
\begin{tabular}{@{}ccccccc@{}}
\toprule
 &  \multicolumn{2}{c}{\ce{(H2O)_{20}}}  &  \multicolumn{2}{c}{\ce{(H2O)_{24}}}  &  \multicolumn{2}{c}{\ce{(H2O)_{28}}}\\ \midrule
$k$       & $N_k$        & $n_k $      & $N_k$         & $n_k$          & $N_k$            & $n_k$           \\ \hline
0       & 10,464    & 94       & 65,312     & 2,739       & 429,312       & 17,888       \\
1       & 155,520   & 1,296    & 1,248,912  & 52,038      & 9,619,296     & 400,804      \\
2       & 622,560   & 5,195    & 6,544,680  & 272,777     & 63,529,368    & 2,647,057    \\
3       & 1,086,960 & 9,058    & 15,906,040 & 662,767     & 200,548,656   & 8,356,194    \\
4       & 975,360   & 8,132    & 20,629,752 & 859,716     & 349,190,688   & 14,549,612   \\
5       & 552,096   & 4,604    & 16,889,736 & 703,739     & 388,503,696   & 16,187,654   \\
6       & 184,320   & 1,541    & 8,730,840  & 363,933     & 283,822,656   & 11,825,944   \\
7       & 12,720    & 106      & 2,664,264  & 111,011     & 137,999,376   & 5,749,974    \\
8       & -         & -        & 354,408    & 14,795      & 41,199,552    & 1,716,648    \\
9       & -         & -        & 7,464      & 321         & 6,802,128     & 283,422      \\
10      & -         & -        & -          & -           & 434,376       & 18,099       \\
11      & -         & -        & -          & -           & 1,152         & 48           \\ \hline
Total   & 3,600,000 & 30,026   & 73,041,408 & 3,043,836   & 1,482,080,256 & 61,753,344   \\ \hline
Approx. &           & (30,000) &            & (3,043,392) &               & (61,753,344) \\ \hline
$S_0$      & 0.75482   & -        & 0.75444    & -           & 0.75417       & -            \\ \bottomrule
\end{tabular}
\end{adjustbox}
\begin{spacing}{1.0}
\caption[Number of possible, $N_k$, and non-isomorphic proton configurations, $n_k$, for \ce{(H2O)_{20}}, \ce{(H2O)_{24}} and \ce{(H2O)_{28}} cages. The number of configurations is split into groups depending on the number of (t1d) dimers in the cage, $k$. Notice that $n_k$ is approximately equal to the total number of configurations divided by the order of the symmetry group of the polyhedron, viz. 120, 24 and 24, for the \ce{(H2O)_{20}}, \ce{(H2O)_{24}} and \ce{(H2O)_{28}} cages, respectively.]{Number of possible, $N_k$, and non-isomorphic proton configurations, $n_k$, for \ce{(H2O)_{20}}, \ce{(H2O)_{24}} and \ce{(H2O)_{28}} cages. The number of configurations is split into groups depending on the number of (t1d) dimers in the cage, $k$. Notice that $n_k$ is approximately equal to the total number of configurations divided by the order of the symmetry group of the polyhedron, viz. 120, 24 and 24, for the \ce{(H2O)_{20}}, \ce{(H2O)_{24}} and \ce{(H2O)_{28}} cages, respectively. The residual entropy of each cage is $S_0=k_B\ln(N_k)/N$, where $N$ is the number of molecules in the cage. The Pauling estimate of the residual entropy,  $S_0=0.752\cdot Nk_B$, is slightly smaller than the one derived from explicit enumeration.}\label{tab:MBE_III_T1}
\end{spacing}
\end{table}