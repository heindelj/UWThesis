\begin{table}[t]
\begin{adjustbox}{width=\columnwidth,center}
\begin{tabular}{@{}lccc@{}}
\toprule
(H2O)3       & \multicolumn{1}{p{3cm}}{\centering MP2/AVQZ//\\MP2/AVQZ} & \multicolumn{1}{p{4cm}}{\centering CCSD(T)/AVQZ//\\MP2/AVQZ} & \multicolumn{1}{p{4cm}}{\centering CCSD(T)/AVQZ//\\CCSD(T)/AVQZ}\\
\hline
\textbf{Total Energy} & \textbf{-15.440}   & \textbf{-15.478}       & \textbf{-15.493}           \\
\hline
1-B          & 0.429              & 0.410                  & 0.346                      \\
2-B          & -13.359            & -13.428                & -13.479                    \\
3-B          & -2.510             & -2.460                 & -2.360           \\ \bottomrule         
\end{tabular}
\end{adjustbox}
\begin{spacing}{1.0}
\caption[The MP2/AVQZ and CCSD(T)/AVQZ 1-, 2-, and 3-body terms for \ce{(H2O)3} at the MP2/AVQZ and CCSD(T)/AVQZ geometries. The notation CCSD(T) /AVQZ//MP2/AVQZ means CCSD(T)/AVQZ energies are calculated the MP2 /AVQZ optimized geometry. The relevant monomer reference energies (from left to right), in a.u., are -76.35191864, -76.36358738, and -76.3635876.]{The MP2/AVQZ and CCSD(T)/AVQZ 1-, 2-, and 3-body terms for \ce{(H2O)3} at the MP2/AVQZ and CCSD(T)/AVQZ geometries. The notation CCSD(T)/AVQZ//MP2/AVQZ means CCSD(T)/AVQZ energies are calculated the MP2/AVQZ optimized geometry. The relevant monomer reference energies (from left to right), in a.u., are -76.35191864, -76.36358738, and -76.3635876.}\label{tab:MBE_I_T4}
\end{spacing}
\end{table}